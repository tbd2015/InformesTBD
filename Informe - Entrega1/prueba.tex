\documentclass{memoria}


\begin{document}

\portada{Informe Entrega 1: Ingeniería de Requerimientos}{Gerson Aguirre Pavez\\Max Chacón Villanueva\\Daniel Gacitúa Vásquez\\Elías González Marincovic \\Nicolás Rozas Sepúlveda}{Profesores: Mauricio Marín Caihuán\\Rodrigo Vásquez Fernández \\ Ayudante: José Orellana}{\today}


\indices

%Introducción (No se incluye en los capítulos del documento)
\chapter*{INTRODUCCIÓN.}
\addcontentsline{indices}{chapter}{INTRODUCCIÓN}

En esta era tecnológica, las redes sociales han formado parte de las relaciones interpersonales, donde se asimila una interacción persona a persona tanto o más natural que en el mundo real. Existen distintas aplicaciones web con distintas funcionalidades para fines específicos dependiendo de los gustos de las personas. Por ejemplo, “Twitter” ofrece una plataforma dinámica de publicaciones con pocos caracteres donde la gente puede expresarse o informarse; “Facebook” permite interactuar con los contactos de manera más amplia, permitiendo publicaciones más extensas, incluyendo fotos, videos, archivos, entre otros; en tanto “Vine” permite subir videos de corta duración para expresar sentimientos, emociones o estados de ánimo.

En esta ocasión, se presenta “BitPhoto”, una plataforma ágil y dinámica que permite a distintos usuarios (profesionales o amateur) cargar fotografías e interactuar entre sus contactos, permitiendo un rápido feedback en sus comentarios. Además, permite localizar de manera geográfica las fotografías cargadas y muestra la información técnica pertinente de cada una de las imágenes, todo esto de manera rápida y actualizada.

Como en todo proceso de desarrollo, se requiere de una planificación previa para organizar los puntos importantes que se deben tomar en cuenta para lograr un resultado exitoso. En un primer acercamiento, se debe conocer y entender las necesidades para buscar una posible solución y de esta manera modelar no solo una solución eficaz, sino también eficiente. En esta primera entrega, la ingeniería de requerimientos será parte fundamental para cumplir con lo anteriormente mencionado, formando la base del desarrollo del producto final.

Para entender las necesidades que se deben cubrir con el software que se desarrollará, se analizarán los requerimientos funcionales y no funcionales, con los cuales se tendrá una idea general de las soluciones que deben plantearse. Posteriormente, se deben asimilar todas estas ideas para comenzar un modelamiento del problema, para esto, se utilizará el modelo de casos de uso y diagrama de clases de análisis, permitiendo una visualización un tanto más concreta de la información obtenida anteriormente.

Luego, en lo que se considera una proyección de lo que será el producto final, se utilizarán los prototipos de la interfaz de usuario, donde se observarán los bocetos de la aplicación web y cómo el usuario va a interactuar con ésta, además de un diseño conceptual de la base de datos para garantizar un funcionamiento óptimo del producto.
 
Todo lo anterior se deberá realizar en un ambiente de trabajo colaborativo del equipo, por lo que se definirán roles de los integrantes y las tareas que éstos deberán realizar. Para que exista una organización y coherencia entre quienes realizan las tareas, es que se trabajará con “Git”, un sistema de control de versiones en el cual se puede observar el trabajo que va haciendo cada integrante del equipo y revisarlo para dar una retroalimentación. Por lo mismo, es importante planificar una estimación de los tiempos en los que se tienen que hacer las distintas tareas y tener un registro para observar el cumplimiento de los deberes y analizar el avance del proyecto en general. Todo lo anterior se realizará con “Trello”, una plataforma interactiva y de fácil usabilidad para el equipo de trabajo, que asemejándose a una carta Gantt, mostrará si el avance en las distintas tareas va de acuerdo a lo propuesto o no.

Con esta entrega se espera tener resuelta la ingeniería de requerimientos, que es quizás, la fase más importante al momento de realizar un proyecto, puesto que constituye la base en la que se trabajará para implementar el resultado final. Con las correcciones que se harán con posterioridad, se comenzará la siguiente fase donde se trabajará la arquitectura para, de esta manera, comenzar a concretar las tareas realizadas en esta oportunidad.   

\capitulo{MARCO TEÓRICO.}


ñandu ñoño murciélago Lorem ipsum ad his scripta blandit partiendo, eum fastidii accumsan euripidis
in, eum liber hendrerit an. Qui ut wisi vocibus suscipiantur, quo dicit ridens
inciderint id. Quo mundi lobortis reformidans eu, legimus senserit definiebas an
eos. 

Eu sit tincidunt incorrupte definitionem, vis mutat affert percipit cu,
eirmod consectetuer signiferumque eu per. In usu latine equidem dolores. Quo no
falli viris intellegam, ut fugit veritus placerat per. Ius id vidit volumus
mandamus, vide veritus democritum te nec, ei eos debet libris consulatu. No mei
ferri graeco dicunt, ad cum veri accommodare. Sed at malis omnesque delicata,
usu et iusto zzril meliore. Dicunt maiorum eloquentiam cum cu, sit summo dolor
essent te. Ne quodsi nusquam legendos has, ea dicit voluptua eloquentiam pro, ad
sit quas qualisque. Eos vocibus deserunt quaestio ei. Blandit incorrupte
quaerendum in quo, nibh impedit id vis, vel no nullam semper audiam. Ei populo
graeci consulatu mei, has ea stet modus phaedrum. Inani oblique ne has, duo et
veritus detraxit. 


\seccion{Tota ludos}

Tota ludus oratio ea mel, offendit persequeris ei vim. Eos jueuhfh sjshajsh dsjkfdhsfk fkdjfksj fkds
dicat oratio partem ut, id cum ignota senserit intellegat. Sit inani ubique
graecis ad, quando graecis liberavisse et cum, dicit option eruditi at duo.
Homero salutatus suscipiantur eum id, tamquam voluptaria expetendis ad sed,
nobis feugiat similique usu ex. Eum hinc argumentum te, no sit percipit
adversarium, ne qui feugiat persecuti. Odio omnes scripserit ad est, ut vidit
lorem maiestatis his, putent mandamus gloriatur ne pro. Oratio iriure rationibus
ne his, ad est corrumpit splendide. Ad duo appareat moderatius, ei falli tollit
denique eos. Dicant evertitur mei in, ne his deserunt perpetua sententiae, ea
sea omnes similique vituperatoribus. 


\subseccion{Ex mel}

Ex mel errem intellegebat comprehensam, vel
ad tantas antiopam delicatissimi, tota ferri affert eu nec. Legere expetenda
pertinacia ne pro, et pro impetus persius assueverit. Ea mei nullam facete,
omnis oratio offendit ius cu. Doming takimata repudiandae usu an, mei dicant
takimata id, pri eleifend inimicus euripidis at. His vero singulis ea, quem
euripidis abhorreant mei ut, et populo iriure vix. Usu ludus affert voluptaria
ei, vix ea error definitiones, movet fastidii signiferumque in qui. Vis
prodesset adolescens adipiscing te, usu mazim perfecto recteque at, assum putant
erroribus mea in. Vel facete imperdiet id, cum an libris luptatum perfecto, vel
fabellas inciderint ut. Veri facete debitis ea vis, ut eos oratio erroribus.
Sint facete perfecto no vel, vim id omnium insolens. Vel dolores perfecto
pertinacia ut, te mel meis ullum dicam, eos assum facilis corpora in. Mea te
unum viderer dolores, nostrum detracto nec in, vis no partem definiebas
constituam. Dicant utinam philosophia has cu, hendrerit prodesset at nam, eos an
bonorum dissentiet.\\



Has ad placerat intellegam consectetuer, no adipisci
mandamus senserit pro, torquatos similique percipitur est ex. Pro ex putant
deleniti repudiare, vel an aperiam sensibus suavitate. Ad vel epicurei
convenire, ea soluta aliquid deserunt ius, pri in errem putant feugiat. Sed
iusto nihil populo an, ex pro novum homero cotidieque. Te utamur civibus
eleifend qui, nam ei brute doming concludaturque, modo aliquam facilisi nec no.
Vidisse maiestatis constituam eu his, esse pertinacia intellegam ius cu. Eos ei
odio veniam, eu sumo altera adipisci eam, mea audiam prodesset persequeris ea.
Ad vitae dictas vituperata sed, eum posse labore postulant id. Te eligendi
principes dignissim sit, te vel dicant officiis repudiandae. Id vel sensibus
honestatis omittantur, vel cu nobis commune patrioque. In accusata definiebas
qui, id tale malorum dolorem sed, solum clita phaedrum ne his. Eos mutat ullum
forensibus ex, wisi perfecto urbanitas cu eam, no vis dicunt impetus. Assum
novum in pri, vix an suavitate moderatius, id has reformidans referrentur. Elit
inciderint omittantur duo ut, dicit democritum signiferumque eu est, ad suscipit
delectus mandamus duo. An harum equidem maiestatis nec. At has veri feugait
placerat, in semper offendit praesent his. Omnium impetus facilis sed at, ex
viris tincidunt ius. Unum eirmod dignissim id quo. 


Sit te atomorum quaerendum neglegentur, his primis tamquam et. Eu quo quot veri
alienum, ea eos nullam luptatum accusamus. Ea mel causae phaedrum reprimique, at
vidisse dolores ocurreret nam.  

\newpage
Esto esta en otra pagina

\capitulo{ENUNCIADO DEL PROBLEMA.}
Asdf

\capitulo{MISIÓN Y VISIÓN DEL PRODUCTO DE SOFTWARE.}

\seccion{MISIÓN.}

Nuestra misión como Threads \& Bits Desarrolladores, organización destinada al desarrollo de aplicaciones web y móviles, es  lograr que las personas puedan compartir y hacer del mundo un lugar más abierto y conectado, ofreciendo a nuestros clientes productos de excelencia y con altos estándares de calidad cumpliendo así con sus expectativas y necesidades. Además, consideramos que nuestros clientes son de gran importancia para la organización, por lo cual, nos hemos propuesto entregar productos innovadores, que otorguen soluciones efectivas, seguras, confiables y de alta calidad para responder a las necesidades cambiantes, y contribuir así, a los objetivos particulares de cada uno de éstos, proporcionando de esta manera altos niveles de satisfacción.

Como organización creemos que “BitPhoto” será un producto que cumpla con las expectativas de nuestros clientes, con nuestra aplicación buscamos otorgar una nueva experiencia a las personas, queremos que las personas compartan sus fotografías, sus trabajos, su arte con el mundo, todo esto de manera cómoda, segura y confiable. El ideal de “BitPhoto” es que el usuario tenga la posibilidad de conectarse y comunicarse con todo el mundo, desde amigos o familiares hasta compañeros de trabajo. En conclusión en Threads \& Bits Desarrolladores queremos que Bitphoto se transforme en la red social, para fotógrafos amateurs y profesionales de más éxito a nivel nacional e internacional.


\seccion{VISIÓN.}


Ser una organización multidisciplinaria con el fin de ofrecer soluciones en diversas áreas, integrando la tecnología y principalmente la informática a éstas. Queremos ser una organización que genere confianza en nuestros clientes, y con la sociedad. Que la organización se caracterice por ser una fuente de inversión para el capital humano, permitiéndonos así crear productos de mejor calidad.

A futuro nos vemos como una organización de prestigio y seriedad en el mercado tanto nacional como internacional, logrando así con el tiempo, posicionarnos como un referente nacional en el desarrollo de soluciones informáticas para diversas áreas y campos de la sociedad actual. Queremos llegar a ser un motor que potencie el desarrollo de aplicaciones en nuestro país y latinoamérica, con alta presencia en el mercado internacional, mediante la utilización de tecnología de punta para la creación de nuestros productos. La visión a mediano o largo plazo es poder captar y cautivar a una amplia variedad de clientes, satisfaciendo sus necesidades de manera íntegra y efectiva. Buscamos trabajar con las grandes, medianas y pequeñas empresas en la implementación de soluciones vanguardistas que satisfagan sus necesidades.

Threads \& Bits Desarrolladores tiene como objetivo a corto plazo satisfacer el mercado nacional con la creaciones de nuestro producto “BitPhoto” el cual creemos que nos permitirá crecer y mejorar de manera continua, permitiéndonos de esta manera ampliar el mercado en el cual nos movemos  y mejorar la calidad de los servicios que ofrecemos a nuestros clientes.

Creemos que “BitPhoto” se transformará en la plataforma donde cada fotógrafo profesional o amateur podrá mostrar su trabajo al mundo y conectarse más por medio de las redes sociales.  También tenemos la visión que “BitPhoto” permitirá a organizaciones de distintas índoles mostrar sus actividades por medio de la fotografía. Y creemos firmemente que “BitPhoto” se transformará en la aplicación más utilizada por las personas para almacenar y compartir sus fotografías con familia, amigos, compañeros de trabajo, etc.


\capitulo{REQUERIMIENTOS DEL SISTEMA.}

\seccion{REQUERIMIENTOS FUNCIONALES.}

\subseccion{VISTA PERFIL DE USUARIO.}

\textbf{- Camera Roll:}

RF1: El usuario puede ver las fotografías organizadas por fecha, escogiendo por la fecha en que fueron tomadas o la fecha en que fueron cargadas a la aplicación.\\

\textbf{- Photo stream:}

RF2: El usuario puede visualizar las fotografías que ha cargado sin importar si éstas se encuentran en álbumes distintos.

RF3: El usuario puede seleccionar una fotografía cargada para revisar en detalle los datos importantes de ésta (etiquetas, personas que se encuentran en la foto, comentarios, datos técnicos de la cámara, lugar, fecha, cuantas veces ha sido vista y cuantas personas han marcado como favorita la fotografía).

RF4: El usuario puede agregar personas en una fotografía cargada.

RF5: El usuario puede agregar tags o etiquetas a una fotografía cargada.

RF6: El usuario puede comentar una fotografía cargada.

RF7: El usuario puede agregar o mover una fotografía de un álbum a otro.\\

\textbf{- Albums:}

RF8: El usuario puede visualizar los distintos álbumes creados con una de las fotografías que contengan como carátula.

RF9: El usuario puede seleccionar un álbum para visualizar todas las fotografías contenidas en éste.

RF10: El usuario puede crear un nuevo álbum ingresando datos importantes como un nombre, una descripción, y una o más fotografías que ya se encuentren cargadas en la aplicación. \\

\textbf{- Map:}

RF11: El usuario visualiza un mapa del mundo donde puede ubicar de manera geográfica las fotografías que ha cargado, guardándose una dirección donde puede observar la fotografía. 

RF12: El usuario puede buscar una dirección en el mapa. \\

\textbf{- Favorites:}

RF13: El usuario puede visualizar las fotografías de otras personas donde ha marcado la opción “favorito”.\\

\textbf{- Recent Activity:}

RF14: El usuario puede ver las últimas interacciones relacionadas con actividad en sus fotos, respuestas a sus comentarios en fotos de otras cuentas, notificaciones de seguimiento (follow y followback) y fotos en las que se ha añadido.

RF15: El usuario puede seleccionar un período de tiempo para ver las interacciones que ocurrieron.\\

\subseccion{VISTA DE PERSONAS.}

\textbf{- Photos from:} 

RF16: El usuario puede  ver entre 1 o 5 fotos de las personas a los cuales está siguiendo.\\

\textbf{- Photo of:} 

RF17: El usuario puede visualizar las fotografías en las que se han marcado personas a las que está siguiendo.\\

\subseccion{VISTA EXPLORAR.}

\textbf{- Recent photos:}

RF18: El usuario puede visualizar las fotos recientemente subidas a la aplicación sin importar si sigue o no a las demás personas.

RF19: El usuario puede marcar la opción “favorito” en cualquiera de las fotografías recientes.

RF20: El usuario puede comentar cualquiera de las fotografías recientes.

RF21: El usuario puede ver toda la información técnica de las fotografías recientes.\\

\textbf{- World Map:}

RF22: El usuario puede ver el mapa mundial de fotografías que han sido subidas con ubicación a la aplicación sin importar si está siguiendo o no a las personas.

RF23: El usuario puede seleccionar una marca en el mapa de un lugar específico (país o ciudad) para ver las fotografías localizadas en esa ubicación.

RF24: El usuario puede buscar una dirección, un país o una ciudad  en el mapa, para visualizar todas las fotografías que se han marcado en dicho lugar.\\

\textbf{- Camera Finder:}

RF25: El usuario puede ver las cámaras más populares con las que se han tomado fotografías subidas a la aplicación.

RF26: El usuario puede ver un listado de las distintas marcas de cámaras que se han utilizado.

RF27: El usuario puede ver las especificaciones técnicas de las distintas cámaras, seleccionando un modelo en el listado de cámaras populares.

RF28: El usuario puede visualizar las fotografías que fueron sacadas con una cámara en específico.\\

\subseccion{VISTA DE BÚSQUEDA.}

\textbf{- Sort:} 

RF29: El usuario puede ordenar la búsqueda por orden de relevancia, fotografías recientes o fotografías interesantes.\\

\textbf{- Search:}

RF30: El usuario visualiza las fotografías relacionadas con las palabras clave de su búsqueda.

RF31: El usuario puede filtrar la búsqueda por las fotografías de todos, por sus propias fotografías, por fotografías de sus contactos o usuarios.\\

\textbf{- License:}

RF32: El usuario puede filtrar la búsqueda, mostrándose solo fotografías con permiso comercial, con permiso de modificación, con ambos permisos o fotografías con cualquier tipo de licencias.\\

\subseccion{VISTA SUBIR IMÁGENES.}

\textbf{- Add:}

RF33: El usuario puede arrastrar una o varias fotografías o buscarlas en los directorios de su equipo para cargarla a la aplicación.\\

\textbf{- Remove:} 

RF34: El usuario puede remover una o varias fotografías que hayan sido seleccionadas para ser cargadas a la aplicación.\\ 

\textbf{- Change name:}

RF35: El usuario puede cambiar el nombre de una o varias fotografías que hayan sido seleccionadas para ser cargadas a la aplicación.\\

\textbf{- Add tags:}

RF36: El usuario puede agregar variadas etiquetas o tags a una o varias fotografías que hayan sido seleccionadas para ser cargadas a la aplicación.\\

\textbf{- Add people:}

RF37: El usuario puede agregar distintas personas en una o varias fotografías que hayan sido seleccionadas para ser cargadas a la aplicación.\\

\textbf{- Add to albums:}

RF38: El usuario puede agregar una o varias fotografías, que hayan sido seleccionadas para ser cargadas a la aplicación, a distintos álbumes sin importar si estos ya ha existen o deben crearse.\\

\textbf{- Owner Settings:}

RF39: El usuario puede escoger la licencia que desee para una o varias fotografías que hayan sido seleccionadas para ser cargadas a la aplicación.\\

\subseccion{VISTA INICIAR SESIÓN.}

RF40: El usuario puede ingresar sus datos de registro para iniciar sesión.

RF41: El usuario puede mantener sus sesión iniciada mediante la opción “Recordar sesión”.\\


\seccion{REQUERIMIENTOS NO FUNCIONALES.} 



\capitulo{CASOS DE USO.} 

\seccion{LISTADO DE CASOS DE USO.}

\seccion{MODELO DE CASOS DE USO.}

\capitulo{DIAGRAMA DE CLASES DE ANÁLISIS.}

\capitulo{PROTOTIPO DE INTERFAZ DE USUARIO.}

\capitulo{BASE DE DATOS.}

\capitulo{GESTIÓN DEL PROYECTO.}
    

\end{document}
